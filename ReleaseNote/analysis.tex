To reconstruct all the particles involved in the reaction we start with the final state particles and go backwards through the reaction chain.

\section{Final state particles}
	The selected final state particle are protons, anti-protons, \piminus, \piplus, \kminus and \kplus mesons.
	For the reconstruction of these particles an ideal tracking was used.
	Ideal tracking means that the hit points caused by a particle track are grouped based on the generated particle information. 
	To achieve a more realistic reconstruction efficiency only particles with at least 4 hits in any inner tracking detector (MVD, STT and GEM)
	are selected.
	The selection criterion is chosen because three hits are needed to define a circle.
	A fourth hit point is then a validation of the track hypothesis.\\
	The particle identification (PID) is also ideal meaning that the true particle gets the probability $P=1$, the others $P=0$. 
	The selection criterion is set to 'best'.\vspace{11pt} \\
	The reconstruction efficiency and the momentum resolution for the final state particle is shown in table \ref{tab:finalstate_recoeff} and figure \ref{fig:finalstate_recoeff}.
	All reconstruction efficiencies are calculated with the MC matched particles.
	
	\begin{table}
		\centering
		\caption{\propose Reconstruction efficiency and momentum resolution for \pbarp $\rightarrow$ \excitedcascade \anticascade. The errors are pure statistical errors.}
		\label{tab:finalstate_recoeff}
		\begin{tabular}{lllll}
			\hline
			final state & N$\left[\%\right]$ & $\sigma_N^{\mt{stat.}}\left[\%\right]$ &$\frac{\sigma p}{p}\left[\%\right]$ & $\sigma_\frac{\sigma p}{p}\left[\%\right]$ \\
			\hline
			\hline
			\piminus & 83.48 & $0.1$& 1.53 & $3\cdot 10^{-3}$\\
			\piplusone(\anticascade) &  80.93& $0.1$& 1.38 & $3\cdot 10^{-3}$ \\
			\piplustwo(\alam) &  83.07& $0.1$&1.49 & $3\cdot 10^{-3}$\\
			\kminus&  78.59& $0.1$& 1.58 & $3\cdot 10^{-3}$\\
			p &  84.39& $0.1$& 1.61 & $4\cdot 10^{-3}$\\
			\antiproton & 78.25 &$0.1$& 1.45 & $4\cdot 10^{-3}$\\\hline
			 
		\end{tabular}
	\end{table}
	
	Table \ref{tab:finalstate_recoeff_cc} shows the reconstruction efficiency and the momentum resolution for the charge conjugate channel.
	
	\begin{table}
		\centering
		\caption{\propose Reconstruction efficiency and momentum resolution for \pbarp $\rightarrow$ \excitedanticascade \cascade. The errors are pure statistical errors.}
		\label{tab:finalstate_recoeff_cc}
		\begin{tabular}{lllll}
			\hline
			final state & N$\left[\%\right]$ & $\sigma_N^{\mt{stat.}}\left[\%\right]$ &$\frac{\sigma p}{p}\left[\%\right]$ & $\sigma_\frac{\sigma p}{p}\left[\%\right]$ \\
			\hline
			\hline
			\piplus &  82.96& $0.1$&  1.54 & $2\cdot 10^{-3}$\\
			\piminusone(\cascade) & 80.40& $0.1$&   1.38 & $2\cdot 10^{-3}$  \\
			\piminustwo(\lam) &  82.69&  $0.1$& 1.49& $3\cdot 10^{-3}$\\
			\kplus& 83.27& $0.1$&  1.58 & $3\cdot 10^{-3}$\\
			p &  80.71& $0.1$&  1.55& $4\cdot 10^{-3}$\\
			\antiproton &  80.93& $0.1$&  1.60 & $4\cdot 10^{-3}$\\\hline
			 
		\end{tabular}
	\end{table}
	
	\begin{figure}
	
		\centering
		\includegraphics[width=0.8\textwidth]{./plots/finalstate/reco_efficiency.pdf}
		\caption{\propose Reconstruction efficiency for final state particles. The x axis shows the particle type. 
				On the y axis the fraction of reconstructed particles is shown.}
		\label{fig:finalstate_recoeff}
	
	\end{figure}
	

	
\section{Reconstruction of $\Lambda$ and $\bar{\Lambda}$}\label{subsec:lambda}
	\subsection*{Selection}
		For the reconstruction of \lam hyperons a proton and a \piminus meson are combined and for the reconstruction 
		of \alam a \antiproton and a \piplus are combined.		 
		After combining the daughter particles a mass cut is performed.
		Only those candidates are chosen which have a mass within a window of $0.3$\massunit symmetric to the nominal 
		\lam mass, i.e., a mass within $m= 1.116\pm 0.15$\massunit.
		
		
		A vertex constraint fit with the PndKinVtxFitter is performed on the selected candidate.
		This means that the tracks of the daughter particles are fitted to a common vertex point.  
		The \chisq and probability distribution of the vertex fit for \lam candidates is shown in figure \ref{fig:lambda_chi2}.
		
		\begin{figure}
			\centering
				\includegraphics[width=0.50\textwidth]{./plots/lambda0/lambda0_chi2.pdf}
				\includegraphics [width=0.50\textwidth]{./plots/lambda0/lambda0_prob.pdf}
			\caption{\propose The upper plot shows the  \chisq distribution and the lower plot shows the probability distribution for the \lam vertex fit.}
			\label{fig:lambda_chi2}
		\end{figure}
		
		In the probability distribution one can see an increasing number of events for probabilities approaching a value of one.
		To understand the origin of this behaviour the vertex fitter was tested with the "poormantrack" algorithm [4]. 
		This algorithm creates simple particle tracks without using any detector information.
        The particle tracks were fitted to a common vertex point with the PndKinVtxFitter.
        These tests have shown that the behaviour of the probability distribution is not caused by the PndKinVtxFitter code. 
        The origin of this behaviour is still under investigation.
		\vspace{11pt}\\
		A mass constraint fit is performed on the fitted candidate.
		For this mass constraint fit the kinematic fitter PndKinFitter is used.
		After using both fitters the selection criterion is set. 
		Only those particles which have a probability larger than $1\%$ in both fitters are selected.
		A scheme which shows how the events are selected can be found in figure \ref{fig:lambda_scheme}. 
		
		\begin{figure}
			\centering
				\includegraphics[width=0.45\textwidth]{./plots/combineLambda0.pdf}
			\caption{\propose Scheme for \lam reconstruction}
			\label{fig:lambda_scheme}
		\end{figure}
		
		If there is more than one candidate left after these cuts, the candidate with the lowest \chisq is chosen.
		
		
	\subsection*{Results}
		In this paragraph the \lam and \alam sample obtained with the chosen selection criteria is presented.
		The mass distributions corresponding to the different cuts are shown in figure \ref{fig:lambda0_massdiffcuts} 
		and figure \ref{fig:antilambda0_massdiffcuts} for \lam and \alam, respectively.
	
		\begin{figure}
			\centering
				\includegraphics[width=1.1\textwidth]{./plots/lambda0/lambda0_m_diffcuts.pdf}
			\caption{\propose Mass distribution of \lam after the mass cut (blue), after the vertex fit cut (red) and after all cuts (black).}
			\label{fig:lambda0_massdiffcuts}
		\end{figure}
			
		\begin{figure}
			\centering
				\includegraphics[width=1.1\textwidth]{./plots/antilambda0/antiLambda0_m_diffcuts.pdf}
			\caption{\propose Mass distribution of \alam after the mass cut (blue), after the vertex fit cut (red) and after all cuts (black).}
			\label{fig:antilambda0_massdiffcuts}
		\end{figure}
		
		The reconstructed mass can be determined by performing a double Gaussian fit on the mass distribution obtained after all cuts.
		The mass distribution and the double Gaussian fit are shown for \lam candidates in figure \ref{fig:lambda0_massfit}.
		
		\begin{figure}
			\centering
				\includegraphics[width=0.8\textwidth]{./plots/lambda0/lambda0_m_masscut2.pdf}
			\caption{\propose Mass distribution (blue histogram) for \lam fitted with a double Gaussian fit (red dashed line). 
			The inner Gaussian fit is shown as blue dashed line and the outer Gaussian fit as black dashed line.}
		
			\label{fig:lambda0_massfit}
		\end{figure}
		
		The peak position of the Gaussian fit is taken as the value of the reconstructed mass.
		The reconstructed masses are $\mt{m}_{\Lambda} = \left(1.1158 \pm 0.0021\right)$\massunit and 
		$\mt{m}_{\bar{\Lambda}} = \left(1.1158 \pm 0.0021\right)$\massunit for \lam and \alam, respectively. 
		Figure \ref{fig:lambda0_pt_vs_pz} shows the transverse momentum versus the longitudinal momentum.
				
		\begin{figure}
			
			\subfigure[]{\includegraphics[width=0.49\textwidth]{./plots/lambda0/lambda0_pt_vs_pz_cut.pdf}}
			\subfigure[]{\includegraphics[width=0.49\textwidth]{./plots/antilambda0/antiLambda0_pt_vs_pz_cut.pdf}}
			\caption{\propose Figure a): transverse versus longitudinal momentum for \lam. Figure b):transverse versus longitudinal momentum for \alam.}
			\label{fig:lambda0_pt_vs_pz}
		
		\end{figure}
		After all cuts the reconstruction efficiency is $40.48\%$ for \lam and $33.42\%$ for \alam.
		The difference in the reconstruction efficiencies for \lam and \alam is caused by the different decay lengths of their mother particles.
		\lam is emitted by the \excitedcascade which has a very short decay length while the decay length of \anticascade is 
		c$\tau = 4.91 \unit{cm}$ \cite{PDG}.
		The decay length of \lam and \alam is $\textnormal{c}\tau = 7.98 \unit{cm}$, so that the final state particles of \alam are produced more downstream 
		than the final state particles of \lam.
		This can be also seen in figure \ref{fig:lambda0_antilambda0_decay_vtx}.
		The final state particles of \alam are produced at the edge of the MVD detector so that the reconstruction efficiency for these particles is reduced.
		An extension of the MVD with two more discs the so-called "Lambda-Discs" might improve the reconstruction efficiencies for \lam and \alam.
		
		\begin{figure}
		
			\centering
			\includegraphics[width=1.\textwidth]{./plots/lambda0/lambda0_decay_vtx.pdf}
			\includegraphics[width=1.\textwidth]{./plots/antilambda0/antiLambda0_decay_vtx.pdf}
			\caption{\propose Decay vertex position of \lam (upper plot) and \alam (lower plot). 
					In both plots the x axis shows the z coordinate (along the beam axis) of the 
					decay vertex while the y axis shows its radial coordinate (the origin of the coordinate system s defined by the primary vertex).
					The black horizontal and vertical lines mark the radial and longitudinal extension of the MVD.}
			\label{fig:lambda0_antilambda0_decay_vtx}
		
		\end{figure}
		
		
		
		
	
\section{Reconstruction of \cascade and \anticascade}
	\subsection*{Selection}
		The reconstruction of \cascade and \anticascade follows a scheme similar to the reconstruction of \lam and \alam.
		For \anticascade \alam and \piplus are recombined, for \cascade in the charge conjugate channel \lam and \piminus.
		In case of the correct selection both the \piminus or \piplus candidate as daughter particle of \lam or \alam, respectively,
		only one \piminus or \piplus candidate remains within the primary particles of the reaction chain, which must be the daughter 
		particle of \cascade or \anticascade, respectively.
		The correct selection of the \lam and \alam daughter pions is assured by the choice of the best fitted \lam and \alam candidates, 
		as described in section \ref{subsec:lambda}.
		The pions associated to the \lam and \alam decay are removed from the pion candidate lists used for the reconstruction of \cascade and \anticascade .
		After combining the daughter particles a mass cut is performed corresponding to a window with a width of $0.3$\massunit symmetric around 
		the \cascade mass $m_{\Xi} = 1.32171$ \massunit \cite{PDG}.
		 
		The fitting scheme is the same as for \lam and \alam and is shown in figure \ref{fig:anticascade_scheme}.
		After the mass cut the daughter particles are fitted to a common vertex with the PndKinVtxFitter.
		The resulting candidates are used to perform the mass constraint fit. 
		
		\begin{figure}
			\centering
				\includegraphics[width=0.50\textwidth]{./plots/combineAntiCascade.pdf}
			\caption{\propose Scheme for \anticascade reconstruction}
			\label{fig:anticascade_scheme}
		\end{figure}
		
		Only those particles are selected which have a probability of more than $1\,\%$ in both fitters. 
		Figure \ref{fig:XiPlus_prob} shows exemplarily the cut on the vertex fit probability for \anticascade.
		
		\begin{figure}
			\centering
				\includegraphics[width=0.50\textwidth]{./plots/Xi/XiPlus_prob.pdf}
			\caption{\propose probability distribution for \anticascade reconstruction.}
			\label{fig:XiPlus_prob}
		\end{figure}
			
		If there is more than one candidate left after all cuts, e.g. due to additional pions produced in secondary interactions, the candidate with the 
		lowest \chisq is chosen.
		
		
	\subsection*{Results}
		The vertex resolution after all cuts is shown in table \ref{tab:XiPlus_vtxres}. 
		
		\begin{table}
			\centering
			\caption{\propose Vertex resolution for \anticascade and \cascade (charge conjugate. channel)}
			\label{tab:XiPlus_vtxres}
			\begin{tabular}{ccc}
				\hline
				position & \anticascade & \cascade(from charge conjugate.) \\\hline
				\hline
				x/cm & $0.052$ & $0.056$\\
				y/cm & $0.052$ & $0.052$\\
				z/cm & $0.19$ & $0.2$\\
				\hline
				    
			\end{tabular}
		\end{table}
		
		It is determined by calculating the full width at half maximum (FWHM) of the distribution.
		The advantage of using this method for calculating the vertex resolution is that the FWHM is independent of the shape of the distribution.
		Figures \ref{fig:xi_vtxres_xy} and \ref{fig:xi_vtxres_z} show the vertex resolution distribution. 
		

		
		\begin{figure}
			\subfigure[Vertex resolution for x coordinate]{\includegraphics[width=0.49\textwidth]{./plots/Xi/XiPlus_vtxres_x.pdf}}
			\subfigure[Vertex resolution for y coordinate]{\includegraphics[width=0.49\textwidth]{./plots/Xi/XiPlus_vtxres_y.pdf}}
			\caption{\propose The left plot shows the vertex resolution in the x coordinate for \anticascade. The right plot shows the vertex
					 resolution of in the y coordinate for \anticascade.}
			\label{fig:xi_vtxres_xy}
			
		\end{figure}
		
		\begin{figure}
			\centering
			\includegraphics[width=0.8\textwidth]{./plots/Xi/XiPlus_vtxres_z.pdf}
			\caption{\propose Vertex resolution of in the z coordinate for \anticascade candidates.}
			\label{fig:xi_vtxres_z}
			
		\end{figure}
	
		The mass distribution obtained with the different cuts is shown in figure \ref{fig:XiPlus_massdiffcuts} and figure \ref{fig:XiMinus_massdiffcuts} 
		for \anticascade and \cascade, respectively.
		The number of events is strongly reduced by the cut on the vertex fit probability. 
		The width of the mass distribution is reduced. 
		
		\begin{figure}
			\centering
				\includegraphics[width=1.1\textwidth]{./plots/Xi/XiPlus_m_diffcuts.pdf}
			\caption{\propose Mass distribution of the \anticascade after after the mass cut (blue), after the vertex fit cut (red) and after all cuts (black).}
			\label{fig:XiPlus_massdiffcuts}
		\end{figure}
		
		\begin{figure}
			\centering		
				\includegraphics[width=1.1\textwidth]{./plots/Xi/XiMinus_m_diffcuts.pdf}
			\caption{\propose Mass distribution of the \cascade after after the mass cut (blue), after the vertex fit cut (red) and after all cuts (black).}
			\label{fig:XiMinus_massdiffcuts}
		\end{figure}
		
		After using all cuts on the mass distribution the reconstructed mass of \cascade and \anticascade can be determined by a double Gaussian fit.
		This is exemplarily shown for the \cascade in figure \ref{fig:XiPlus_massfit}.
		
		\begin{figure}
			\centering
				\includegraphics[width=0.8\textwidth]{./plots/Xi/XiPlus_m_masscut.pdf}
			\caption{\propose The plot shows the mass distribution (blue histogram) after all cuts. 
					A double Gaussian fit (red dashed line) is performed to determine the mean reconstructed mass for the \anticascade. 
					The inner Gaussian fit is shown as blue dashed line and the outer Gaussian fit as black dashed line.}
			\label{fig:XiPlus_massfit}
		\end{figure}
		The result of the mass fit is for \anticascade ${\mt{m} = \left( 1.322 \pm 0.004\right)}$ \massunit 
		and for \cascade \\ ${\mt{m} = \left( 1.322 \pm 0.004\right)}$\massunit.
		The two dimensional momentum distribution for \anticascade and \cascade is shown in figure \ref{fig:XiPlus_pt_vs_pz} 
		
		\begin{figure}
			\subfigure[\anticascade]{\includegraphics[width=0.49\textwidth]{./plots/Xi/XiPlus_pt_vs_pz_cut.pdf}}
			\subfigure[\cascade]{\includegraphics[width=0.49\textwidth]{./plots/Xi/XiMinus_pt_vs_pz_cut.pdf}}
			\caption{\propose The plots shows the transverse versus the longitudinal momentum for \anticascade and \cascade after vertex fit and mass fit.}
			\label{fig:XiPlus_pt_vs_pz}
		
		\end{figure}
		
		The reconstruction efficiency for \anticascade is $18.4\%$ and for \cascade $18.6\%$.
		
	
	
	

\section{Reconstruction of \excitedcascade and \excitedanticascade}
		\subsection*{Selection}

		For the reconstruction of \excitedcascade one combines the \lam candidate with the \kminus meson candidate and for \excitedanticascade \alam and \kplus, using the
		fit candidates \lam and \alam.
		After the combination of the particles a mass cut with a width of $0.3$\massunit symmetric to the nominal \excitedcascade mass is performed. 
		The daughter particles are  then fitted to a common vertex point with the PndKinVtxFitter.
		Only those candidates for \excitedcascade (\excitedanticascade) are selected which have a fit probability of more then $1\%$.
		The selection scheme is shown in figure \ref{fig:excitedcascade_scheme}. 
		
		\begin{figure}
			\centering
				\includegraphics[width=0.50\textwidth]{./plots/combineExcitedCascade.pdf}
			\caption{\propose Scheme for \excitedcascade reconstruction}
			\label{fig:excitedcascade_scheme}
		\end{figure}
		
		
		The probability distribution for the vertex fit is shown in figure \ref{fig:xi1820_prob}.
		Again the distribution is not flat but increases for values close to one. 
		
		\begin{figure}
			\centering
			\includegraphics[width=1.\textwidth]{./plots/Xi1820/XiMinus1820_prob.pdf}
			\caption{\propose probability distribution of the kinematic vertex fit for \excitedcascade candidates.}
			\label{fig:xi1820_prob}
		\end{figure}
		
		If there is more than one particle the fit candidate with the smallest \chisq is chosen.
		
	\subsection*{Results}
	The vertex resolution for \excitedcascade and \excitedanticascade is summarized in table \ref{tab:Xi1820_vtxres}.
	
	\begin{table}
		\centering
		\caption{\propose Vertex resolution for \excitedcascade and \excitedanticascade.}
		\label{tab:Xi1820_vtxres}
		\begin{tabular}{ccc}
			\hline
			position & \excitedcascade & \excitedanticascade (from charge conjugate.) \\
			\hline
			\hline
			x/cm & 0.028 & 0.028\\
			y/cm & 0.028 & 0.028\\
			z/cm & 0.1 & 0.1\\
			\hline
			 
		\end{tabular}
	\end{table}
	
	Here again the vertex resolution is measured using the FWHM of the distribution. 
	This is shown for \excitedcascade in figure \ref{fig:Xi1820_vtxxy} and \ref{fig:Xi1820_vtxz}.
	
	
	
	 \begin{figure}
		\centering
		\subfigure[Vertex resolution of the x coordinate for \excitedcascade.]{\includegraphics[width=0.49\textwidth]{./plots/Xi1820/XiMinus1820_vtxres_x.pdf}}
		\subfigure[Vertex resolution of the y coordinate for \excitedcascade.]{\includegraphics[width=0.49\textwidth]{./plots/Xi1820/XiMinus1820_vtxres_y.pdf}}
		\caption{\propose Figure a) shows the vertex resolution for the x coordinate and figure b) for the y coordinate of \excitedcascade}
		\label{fig:Xi1820_vtxxy}
	\end{figure}
	
	\begin{figure}
		\centering
		\includegraphics[width=0.6\textwidth]{./plots/Xi1820/XiMinus1820_vtxres_z.pdf}
		\caption{\propose Vertex resolution in the z coordinate for \excitedcascade.}
		\label{fig:Xi1820_vtxz}
	\end{figure}
	
	After performing both fits and cut on the probability values, the mass for \excitedcascade and \excitedanticascade
	can be determined by fitting with a double Gaussian function. 
	Figure \ref{fig:xi1820_mass_diffcuts} shows the mass distribution for both particles after each cut.
	
	\begin{figure}
		\centering
		\includegraphics[width=1.\textwidth]{./plots/Xi1820/XiMinus1820_m_diffcuts.pdf}
		\caption{\propose Mass distribution for \excitedcascade after the mass cut in blue and after the vertex fit probability cut in red.}
		\label{fig:xi1820_mass_diffcuts}
	
	\end{figure}
	As an example the mass fit is shown for the \excitedcascade in figure \ref{fig:xi1820_massfit}. 
	
	\begin{figure}
		\centering
		\includegraphics[width=1.\textwidth]{./plots/Xi1820/XiMinus1820_m_masscut.pdf}
		\caption{\propose Mass distribution (blue histogram) after all cuts for \excitedcascade. The performed double Gaussian fit is shown as the red dashed line.
		The inner Gaussian fit is shown as blue dashed line and the outer Gaussian fit as black dashed line.}
		\label{fig:xi1820_massfit}
	\end{figure}
	The mass value for the \excitedcascade is fitted to $m_{\Xi^{*}} = \left(1.823 \pm 0.014)\right)$ \massunit
	 and for \excitedanticascade to $m_{\bar{\Xi}^{*}} = \left(1.823 \pm 0.014\right)$ \massunit.
	These values are close to the input value.
	Figure \ref{fig:xi1820_pt_vs_pz} shows the two-dimensional momentum distribution of $p_\mt{t}$ versus $p_\mt{z}$.
	
	\begin{figure}
		\centering
		\subfigure[\excitedcascade]{\includegraphics[width=0.49\textwidth]{./plots/Xi1820/XiMinus1820_pt_vs_pz_cut.pdf}}
		\subfigure[\excitedanticascade]{\includegraphics[width=0.49\textwidth]{./plots/Xi1820/XiPlus1820_pt_vs_pz_cut.pdf}}
		\caption{\propose Both plots show the longitudinal versus the transverse momentum of the excited cascade baryon and its antiparticle, respectively.}
		\label{fig:xi1820_pt_vs_pz}
	\end{figure}
	
	The reconstructed distributions are in good agreement with the distribution obtained for the generated events which are 
	shown in figure \ref{fig:MC_xi_pt_vs_pz} (b).
	
	
	
\newpage
\section{Reconstruction of the whole reaction chain}

	\subsection*{Selection}
	
	To reconstruct the whole reaction chain \excitedcascade and \anticascade are combined.
	This is also done with \excitedanticascade and \cascade for the charge conjugated channel.
	For this reconstruction the event selection is done with an exclusive method.
	The resulting four-momentum vector of both daughter particles --  here \excitedcascade 
	and \anticascade and their charge conjugate particles -- is fitted with the constraint to match to the initial four momentum vector  

	\begin{center}
		\begin{equation}\nonumber
			\left(\mt{p}_x,\, \mt{p}_y,\, \mt{p}_z,\, \mt{E} \right) = \left(0,\, 0,\, 4.6,\, 5.63 \right) \unit{GeV}
		\end{equation}
	\end{center}
	of the \pbarp entrance channel.	
	This fit is performed with the PndKinFitter.
	After the four-momentum fit only those candidates are selected which have a probability of more than $1\%$.
	The probability is shown in figure \ref{fig:xisys_prob}. 
	The red line denotes the cut value.

	\begin{figure}
		\centering
		\includegraphics[width=0.7\textwidth]{./plots/pbarp/XiSys_prob.pdf}
		\caption{\propose 4-constraint fit probability. The red line denotes the cut value of $1\%$.}
		\label{fig:xisys_prob}
	\end{figure}
	
	The selection scheme is shown in figure \ref{fig:fourconstraintfit}
	 
	\begin{figure}
		\centering
			\includegraphics[width=0.50\textwidth]{./plots/combineCascadeSys.pdf}
		\caption{\propose Scheme for the reconstruction of the whole reaction chain.}
		\label{fig:fourconstraintfit}
	\end{figure}
	
	\subsection*{Results}
	
	The obtained reconstruction efficiency for all intermediate state particles is shown in table \ref{tab:intermediatestate_efficiency}
	and table \ref{tab:intermediatestate_efficiency_cc}.
	
	\begin{table}
		\centering
		\caption{\propose reconstruction efficiency for intermediate state particles for \pbarp $\rightarrow$ \excitedcascade \anticascade}
		\label{tab:intermediatestate_efficiency}
		
		\begin{tabular}{lcc}
		
			\hline
			particle & reco. efficiency [$\%$] & dp/p [$\%$] \\\hline
			\hline
			\lam & 40.5&   1.40 \\
			\alam & 33.4&   1.49\\
			\anticascade & 18.4&   1.29\\
			\excitedcascade & 32.0&   2.68 \\
			\excitedcascade \anticascade system & 4.7&   1.03\\\hline
			 	
		\end{tabular}
	\end{table}
	
		\begin{table}
		\centering
		\caption{\propose Reconstruction efficiency for intermediate state particles for \pbarp $\rightarrow$ \excitedanticascade \cascade}
		\label{tab:intermediatestate_efficiency_cc}
		
		\begin{tabular}{lcc}
		
			\hline
			particle & reco. efficiency [$\%$] & dp/p [$\%$] \\\hline
			\hline
			\lam & 32.8 &   1.44 \\
			\alam & 40.8 &   1.46\\
			\cascade & 18.6&   2.30\\
			\excitedanticascade & 33.2&   1.31\\
			\excitedanticascade \cascade system & 4.9&   1.03\\\hline
			 	
		\end{tabular}
	\end{table}
	The total numbers of intermediate state particles in \mychannel and \myccchannel are shown in table \ref{tab:nsig} and \ref{tab:nsig_cc}. 
	
	\begin{table}
		\centering
		\caption{number of intermediate state particles for \mychannel including the branching ratios BR(\decay{\lam}{\piminus}{p}) $=0.639$, BR(\decay{\cascade}{\lam}{\piminus})
		 $= 0.999$ and BR(\decay{\excitedcascade}{\lam}{\kminus}) $ =1$.}
		 \label{tab:nsig}
		 
		 \begin{tabular}{lc}
		 	\hline
		 	particle & $N_{\mt{sig}}$ \\\hline
		 	\hline
		 	\lam & 387,993 \\
		 	\alam & 320,316 \\
		 	\anticascade & 176,840 \\
		 	\excitedcascade & 306,955 \\
		 	\excitedcascade \anticascade &  28,714\\\hline
		 	
		 \end{tabular}
	\end{table}
	 
	 \begin{table}
		\centering
		\caption{number of intermediate state particles for \mychannel including the branching ratios BR(\decay{\alam}{\piplus}{\antiproton}) $=0.639$, BR(\decay{\anticascade}{\alam}{\piplus})
		 $= 0.999$ and BR(\decay{\excitedanticascade}{\alam}{\kplus}) $=1$.}
		 \label{tab:nsig_cc}
		 
		 \begin{tabular}{lc}
		 	\hline
		 	particle & $N_{\mt{sig}}$ \\\hline
		 	\hline
		 	\lam & 314,807 \\
		 	\alam & 390,660 \\ 
		 	\cascade & 173,510 \\ 
		 	\excitedanticascade & 318,450 \\
		 	\excitedanticascade \cascade &  29,811\\\hline 
		 	
		 \end{tabular}
	\end{table}
	
	 
	
	These numbers are including the branching ratios for the single decays from \cite{PDG}: BR(\decay{\lam}{\piminus}{p}) $=0.639$ 
	and BR(\decay{\cascade}{\lam}{\piminus}) $= 0.999$. 
	For \excitedcascade the branching ratio is assumed to be BR(\decay{\excitedcascade}{\lam}{\kminus}) $=1$. 
	
	Figure \ref{fig:reco_Dalitzplot} shows the Dalitz plot for the \anticascade, \lam and \kminus final states after the reconstruction before the 4-C kinematic fit. 
	Figure \ref{fig:reco_Dalitzplot} should be compared with the Dalitz plot of the generated particles shown in figure \ref{fig:eventgeneration_Dalitz}
	in order to assess the quality of the reconstruction procedure.
	
	\begin{figure}
		\centering
		\includegraphics[width=0.8\textwidth]{./plots/pbarp/Dalitzplot_reco.pdf}
		\caption{\propose Dalitz plot for reconstructed \anticascade\lam\kminus final state.}
		\label{fig:reco_Dalitzplot}
	
	\end{figure}
	
	