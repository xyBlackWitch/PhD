For background studies 15 million events are simulated with the Dual Parton Model based generator DPM.
To make the selected events of background and signal comparable a scaling factor is needed.
This scaling factor can be calculated with the number of generated events and the cross section of signal and background.
\begin{equation}
		B = \frac{N_\mt{sig}/\left(\sigma_\mt{sig} \cdot \mt{Br}\right)}{N_\mt{bg}/\sigma_\mt{bg}},
\label{eq:bg_scaling}
\end{equation}
where $N_\mt{sig}$ is the number of generated signal events, $N_\mt{bg}$ the number of generated background events and Br the 
branching ratio. The cross section are given by $\sigma_\mt{sig} = 1\,\mu\mt{b}$ and $\sigma_\mt{bg} = 50\unit{mb}$ \cite{PANDAphysics2009}.
The scaling factor is different for each particle.
In table \ref{tab:bg_scaling} the scaling factors are shown for the channel \pbarpSystem $\rightarrow$ \excitedcascade \anticascade.
The scaling factors for the c.c. channel are the same. 

\begin{table}
	\centering
	\caption{Scaling factor $B$ of each particle type for the channel \pbarpSystem $\rightarrow$ \excitedcascade \anticascade.}
	\label{tab:bg_scaling}
	\begin{tabular}{cc}
		\hline
		 Particle & Scaling factor \\
		\hline
		\hline
		&  \\
		\lam & 7,828.16\\
		\anticascade & 7,837.01 \\
		\excitedcascade & 7,837.01\\
		\excitedcascade \anticascade & 12,269.89\\
		\hline
		 
	 \end{tabular}
\end{table}

All background events are reconstructed like the signal events. 
The number of reconstructed Background events is shown in table \ref{tab:bg_reco_without_scaling}.

\begin{table}
	\centering
	\caption{Number of reconstructed background events for \pbarpSystem $\rightarrow$ \excitedcascade \anticascade }
	\label{tab:bg_reco_without_scaling}
	\begin{tabular}{lc}
		\hline
		Particle & $N_\mt{bg}$ \\
		\hline
		\hline
		&\\
		%\pbarpSystem $\rightarrow$ \excitedcascade \anticascade &\\
		\lam & 264,142\\
		\alam & 124,068\\
		\anticascade & 3,062\\
		\excitedcascade & 298\\
		\excitedcascade \anticascade & 0\\
		\hline
		 
		%\pbarpSystem $\rightarrow$ \cascade \excitedanticascade & \\
		%\lam & \\
		%\alam & \\
		%\cascade & \\
		%\excitedanticascade & \\
		%\excitedanticascade \cascade &\\
		%\hline
		  
	\end{tabular}
\end{table}
Multiplying the number of background events with the corresponding scaling factor make them comparable with the number of signal events.
The comparison between signal and background events is shown in table \ref{bg_compared_reco_with_scaling}.
The significance is given by
\begin{equation}
	\nonumber
	S = \frac{N^{2}_\mt{sig}}{N_\mt{sig}+N_\mt{bg}},
\end{equation}
where $N_\mt{bg}$ is scaled with $B$.

\begin{table}
	\centering
	\caption{\propose The number of background events compared to the number of signal events for \pbarpSystem $\rightarrow$ \excitedcascade \anticascade.}
	\label{bg_compared_reco_with_scaling}
	\begin{tabular}{lccc}
		\hline
		Particle & $N_\mt{sig}$ & $N_\mt{bg} \cdot B$ & $S$\\
		\hline
		\hline
		& & &\\
		%\pbarpSystem $\rightarrow$ \excitedcascade \anticascade & & &\\
		\lam & 786,243 &$ 264,142 \cdot B_{\Lambda^0}$& 298.85\\
		\alam & 711,820 & $124,068 \cdot B_{\bar{\Lambda}^0}$ & 521.32\\
		\anticascade & 302,681 & $3,062 \cdot B_{\bar{\Xi}}$ & 3770.26\\
		\excitedcascade &490,672  & $298 \cdot B_{\Xi\left(1820\right)}$& 85,191.23\\
		\excitedcascade \anticascade &  74523 & 0 & \\
		\hline
		 
		%\pbarpSystem $\rightarrow$ \cascade \excitedanticascade & & &\\
		%\lam & & &\\
		%\alam & & &\\
		%\cascade & & &\\
		%\excitedanticascade & & &\\
		%\excitedanticascade \cascade & & &\\
		%\hline
		  
	\end{tabular}
\end{table}
The number of simulated background events is too small to make a useful comparison between the number of signal and background events.
Further background studies for this reaction chain and its charged conjugated channel have to be done as a next step.
	