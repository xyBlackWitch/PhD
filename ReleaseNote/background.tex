For background studies 15 million events are simulated with the Dual Parton Model based generator DPM.
To make the selected events of background and signal comparable a scaling factor is needed.
This scaling factor can be calculated with the number of generated events and the cross section of signal and background.
\begin{equation}
		B = \frac{N_\mt{sig}/\left(\sigma_\mt{sig}\right)}{N_\mt{bg}/\sigma_\mt{bg}},
\label{eq:bg_scaling}
\end{equation}
where $N_\mt{sig}$ is the number of generated signal events and $N_\mt{bg}$ the number of generated background events.
The cross section are given by $\sigma_\mt{sig} = 1\,\mu\mt{b}$ and $\sigma_\mt{bg} = 50\unit{mb}$ \cite{PANDAphysics2009}.
The scaling factor for the channel \pbarpSystem $\rightarrow$ \excitedcascade \anticascade is $B=5000$.
%In table \ref{tab:bg_scaling} the scaling factor is shown for the channel \pbarpSystem $\rightarrow$ \excitedcascade \anticascade.
The scaling factor for the c.c. channel is the same. 

%\begin{table}
%	\centering
%	\caption{Scaling factor $B$ of each particle type for the channel \pbarpSystem $\rightarrow$ \excitedcascade \anticascade.}
%	\label{tab:bg_scaling}
%	\begin{tabular}{cc}
%		\hline
%		 Particle & Scaling factor \\
%		\hline
%		\hline
%		&  \\
%		\lam & 7,828.16\\
%		\anticascade & 7,837.01 \\
%		\excitedcascade & 7,828.16\\
%		\excitedcascade \anticascade & 12,269.89\\
%		\hline
%		 
%	 \end{tabular}
%\end{table}

All background events are reconstructed like the signal events. 
The number of reconstructed Background events is shown in table \ref{tab:bg_reco_without_scaling}.

\begin{table}
	\centering
	\caption{Number of reconstructed background events for \pbarpSystem $\rightarrow$ \excitedcascade \anticascade }
	\label{tab:bg_reco_without_scaling}
	\begin{tabular}{lc}
		\hline
		Particle & $N_\mt{bg}$ \\
		\hline
		\hline
		&\\
		%\pbarpSystem $\rightarrow$ \excitedcascade \anticascade &\\
		\lam & 264,142\\
		\alam & 124,068\\
		\anticascade & 3,062\\
		\excitedcascade & 298\\
		\excitedcascade \anticascade & 0\\
		\hline
		 
		%\pbarpSystem $\rightarrow$ \cascade \excitedanticascade & \\
		%\lam & \\
		%\alam & \\
		%\cascade & \\
		%\excitedanticascade & \\
		%\excitedanticascade \cascade &\\
		%\hline
		  
	\end{tabular}
\end{table}
Multiplying the number of background events with the scaling factor make them comparable with the number of signal events.
The comparison between signal and background events is shown in table \ref{bg_compared_reco_with_scaling}.
The significance is given by
\begin{equation}
	S = \frac{N^{2}_\mt{sig}}{N_\mt{sig}+N_\mt{bg}},
	\label{eq:significance}
\end{equation}
where $N_\mt{bg}$ is scaled with $B$.

\begin{table}
	\centering
	\caption{\propose The number of background events scaled with $B$ compared to the number of signal events for \pbarpSystem $\rightarrow$ \excitedcascade \anticascade.
		The significance is calculated with equation \ref{eq:significance}.}
	\label{bg_compared_reco_with_scaling}
	\begin{tabular}{lccc}
		\hline
		Particle & $N_\mt{sig}$ & $N_\mt{bg} \cdot B$ & $S$\\
		\hline
		\hline
		& & &\\
		%\pbarpSystem $\rightarrow$ \excitedcascade \anticascade & & &\\
		\lam & 786,243 &$ 1.321 \cdot 10^{9}$& 467.68\\
		\alam & 711,820 & $620.341 \cdot 10^{6}$ & 815.85\\
		\anticascade & 302,681 & $15.31 \cdot 10^{6}$ & 5,868.03\\
		\excitedcascade &490,672  & $1.49 \cdot 10^{6}$& 121,544.21\\
		\excitedcascade \anticascade &  74,523 & 0 & < 69,837.37\\
		\hline
		 
		%\pbarpSystem $\rightarrow$ \cascade \excitedanticascade & & &\\
		%\lam & & &\\
		%\alam & & &\\
		%\cascade & & &\\
		%\excitedanticascade & & &\\
		%\excitedanticascade \cascade & & &\\
		%\hline
		  
	\end{tabular}
\end{table}
Because there are no background events left, it is only possible to estimate a upper limit for the significance.
For one background event which is scaled with $B$ the significance is less than 69,837.37.
Further background studies for this reaction chain and its charged conjugated channel have to be done as a next step.
	