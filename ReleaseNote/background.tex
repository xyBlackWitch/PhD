For background studies 15 million events are simulated with the Dual Parton Model based generator DPM.
To make the events of background and signal comparable after the selection, a scaling factor is needed.
This scaling factor can be calculated with the number of generated events and the cross section of signal and background.
\begin{equation}
		B = \frac{N_\mt{sig}/\left(\sigma_\mt{sig} \cdot \mt{Br}\right)}{N_\mt{bg}/\sigma_\mt{bg}},
\label{eq:bg_scaling}
\end{equation}
where $N_\mt{sig}$ is the number of generated signal events, $N_\mt{bg}$ the number of generated background events and Br the 
branching ratio. The cross section are given by $\sigma_\mt{sig} = 1\,\mu\mt{b}$ and $\sigma_\mt{bg} = 50\unit{mb}$ \cite{PANDAphysics2009}.
The scaling factor is different for each particle.
In table \ref{tab:bg_scaling} the scaling factors are shown for the channel \pbarpSystem $\rightarrow$ \excitedcascade \anticascade.
The scaling factors for the c.c. channel are the same. 

\begin{table}
	\centering
	\caption{Scaling factor $B$ of each particle type for the channel \pbarpSystem $\rightarrow$ \excitedcascade \anticascade.}
	\label{tab:bg_scaling}
	\begin{tabular}{cc}
		\hline
		 Particle & Scaling factor \\
		\hline
		\hline
		&  \\
		\lam & 7,828.16\\
		\anticascade & 7,837.01 \\
		\excitedcascade & 7,837.01\\
		\excitedcascade \anticascade & 12,269.89\\
		\hline
		 
	 \end{tabular}
\end{table}

All background events are reconstructed in the same way as the signal events. The number of reconstructed Background events is 
shown in table \ref{tab:bg_reco_without_scaling}.

\begin{table}
	\centering
	\caption{Number of reconstructed background events}
	\label{tab:bg_reco_without_scaling}
	\begin{tabular}{lc}
		\hline
		Particle & $N_\mt{bg}$ \\
		\hline
		\hline
		&\\
		\pbarpSystem $\rightarrow$ \excitedcascade \anticascade &\\
		\lam & \\
		\alam & \\
		\anticascade & \\
		\excitedcascade & \\
		\excitedcascade \anticascade & \\
		\hline
		\pbarpSystem $\rightarrow$ \cascade \excitedanticascade & \\
		\lam & \\
		\alam & \\
		\cascade & \\
		\excitedanticascade & \\
		\excitedanticascade \cascade &\\
		\hline
		  
	\end{tabular}
\end{table}
To compare the number of signal and background events, the number of background events have to be muliplied with the scaling factor for each particle.
The comparison between signal and background events is shown in table \ref{bg_compared_reco_with_scaling}.

\begin{table}
	\centering
	\caption{\propose The number of background events compared to the number of signal events}
	\label{bg_compared_reco_with_scaling}
	\begin{tabular}{lcc}
		\hline
		Particle & $N_\mt{sig}$ & $N_\mt{bg} \cdot B$ \\
		\hline
		\hline
		& &\\
		\pbarpSystem $\rightarrow$ \excitedcascade \anticascade & &\\
		\lam & & \\
		\alam & & \\
		\anticascade & & \\
		\excitedcascade & & \\
		\excitedcascade \anticascade & & \\
		\hline
		\pbarpSystem $\rightarrow$ \cascade \excitedanticascade & & \\
		\lam & & \\
		\alam & & \\
		\cascade & & \\
		\excitedanticascade & & \\
		\excitedanticascade \cascade & &\\
		\hline
		  
	\end{tabular}
\end{table}
	