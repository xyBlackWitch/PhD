For background studies 15 million events have been simulated with the Dual Parton Model based generator DPM.
To compare the number of selected events of background and signal, a scaling factor is needed.
This scaling factor can be calculated with the number of generated events and the cross section of signal and background.
\begin{equation}
		B = \frac{N^{\mt{gen}}_\mt{sig}/\sigma_\mt{sig}}{N^{\mt{gen}}_\mt{bg}/\sigma_\mt{bg}},
\label{eq:bg_scaling}
\end{equation}
where $N^\mt{gen}_\mt{sig}$ is the number of generated signal events and $N^\mt{gen}_\mt{bg}$ the number of generated background events.
The signal and background cross sections are given by $\sigma_\mt{sig} = 1\,\mu\mt{b}$ and $\sigma_\mt{bg} = 60\unit{mb}$ \cite{PANDAphysics2009}, respectively.
The scaling factor is $B=6000$ for the channel \pbarp $\rightarrow$ \excitedcascade \anticascade.
%In table \ref{tab:bg_scaling} the scaling factor is shown for the channel \pbarp $\rightarrow$ \excitedcascade \anticascade.
The scaling factor for the c.c. channel is the same. 
This means that the number of reconstructed events in the background sample surviving all cuts applied to the reconstruction of the signal events 
has to be multiplied by a factor 6000 in order to deduce the achieved signal-to-background ratio. 

%\begin{table}
%	\centering
%	\caption{Scaling factor $B$ of each particle type for the channel \pbarp $\rightarrow$ \excitedcascade \anticascade.}
%	\label{tab:bg_scaling}
%	\begin{tabular}{cc}
%		\hline
%		 Particle & Scaling factor \\
%		\hline
%		\hline
%		&  \\
%		\lam & 7,828.16\\
%		\anticascade & 7,837.01 \\
%		\excitedcascade & 7,828.16\\
%		\excitedcascade \anticascade & 12,269.89\\
%		\hline
%		 
%	 \end{tabular}
%\end{table}

All background events are subject to the same reconstruction procedure including all cuts for the signal events. 
The number of reconstructed background events is shown in table \ref{tab:bg_reco_without_scaling}.

\begin{table}
	\centering
	\caption{Number of reconstructed particles in the background sample for \newline \pbarp $\rightarrow$ \excitedcascade \anticascade }
	\label{tab:bg_reco_without_scaling}
	\begin{tabular}{lc}
		\hline
		Particle & $N_\mt{bg}$ \\
		\hline
		\hline
		&\\
		%\pbarp $\rightarrow$ \excitedcascade \anticascade &\\
		\lam & 264,142\\
		\alam & 124,068\\
		\anticascade & 3,062\\
		\excitedcascade & 298\\
		\excitedcascade \anticascade & 0\\
		\hline
		 
		%\pbarp $\rightarrow$ \cascade \excitedanticascade & \\
		%\lam & \\
		%\alam & \\
		%\cascade & \\
		%\excitedanticascade & \\
		%\excitedanticascade \cascade &\\
		%\hline
		  
	\end{tabular}
\end{table}
The comparison between signal and background events is shown in table \ref{bg_compared_reco_with_scaling}.
The significance is given by
\begin{equation}
	S = \frac{N_\mt{sig}}{\sqrt{N_\mt{sig}+N_\mt{bg}\cdot B}}.
	\label{eq:significance}
\end{equation}

\begin{table}
	\centering
	\caption{\propose The number of background events scaled with factor $B$ compared to the number of signal events for \pbarp $\rightarrow$ \excitedcascade \anticascade.
		The significance is calculated with equation \ref{eq:significance}.}
	\label{bg_compared_reco_with_scaling}
	\begin{tabular}{lccc}
		\hline
		Particle & $N_\mt{sig}$ & $N_\mt{bg} \cdot B$ & $S$\\
		\hline
		\hline
		& & &\\
		%\pbarp $\rightarrow$ \excitedcascade \anticascade & & &\\
		\lam & 607,188 &$ 1.585 \cdot 10^{9}$& 19.74\\
		\alam & 501,277 & $ 744.408 \cdot 10^{6}$ & 26.08\\
		\anticascade & 277,022 & $ 18.372 \cdot 10^{6}$ & 70.04\\
		\excitedcascade &480,368  & $ 1.788 \cdot 10^{6}$& 325.05\\
		\excitedcascade \anticascade &  70,394 & < 6000 & > 262.62\\
		\hline
		 
		%\pbarp $\rightarrow$ \cascade \excitedanticascade & & &\\
		%\lam & & &\\
		%\alam & & &\\
		%\cascade & & &\\
		%\excitedanticascade & & &\\
		%\excitedanticascade \cascade & & &\\
		%\hline
		  
	\end{tabular}
\end{table}
Because none of the background events survives the cuts applied to reconstruct the signal, it is only possible to estimate a lower limit for the significance.
For one background event scaled by the factor $B$ the significance is at least 262.62.
The signal-to-background ratio is for this estimation
\begin{equation}
	\nonumber
	\frac{N_\mt{sig}}{N_\mt{bg}} = \frac{74,523} {6000} = 11.37 : 1
\end{equation}
The values for the significance are unrealistically high since they are obtained from ideal tracking and ideal particle identification.
For the final numbers one have to use a more realistic software.
Additionally for a more precise statement more background studies have to be done.
 
	