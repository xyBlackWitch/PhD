Understanding the excitation pattern of baryons is indispensable for a deep insight into the mechanism of non-perturbative QCD. 
Up to now only the nucleon excitation spectrum has been subject to systematic experimental studies while very little is known 
on excited states of double or triple strange baryons.

\noindent In studies of antiproton-proton collisions the \panda experiment is well-suited for a comprehensive baryon spectroscopy program
in the multi-strange and charm sector. 
A large fraction of the inelastic \pbarp cross section is associated to final states with a baryon-antibaryon pair together with 
additional mesons, giving access to excited states both in the baryon and the antibaryon sector.

\noindent In the present study we focus on excited \cascade states. For final states containing a \cascade\anticascade pair cross sections 
up to the order of $\mu$b are expected, corresponding to production rates of $\sim 10^6/$d at a Luminosity $L=10^{31} \unit{cm}^{-2} \unit{s}^{-1}$
 ($5\%$ of the full value).
A strategy to study the excitation spectrum of \cascade baryons in antiproton-proton collisions will be discussed. The reconstruction of 
reactions of the type \pbarp $\rightarrow \Xi^{-*}$\anticascade (and their charge conjugate) with the \panda detector will be presented 
based on a selected exemplary reaction and decay channel.