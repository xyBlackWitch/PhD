One of the main goals of non-perturbative QCD is the understanding of the excited baryon spectrum. 
The PANDA experiment is well-suited for a comprehensive baryon spectroscopy program. 
A large fraction of channels produced in \pbarpSystem collisions are resulting in a baryon- antibaryon pair in the final state.\\
Antiproton-proton collisions allow also the study of baryon-antibaryon final states in their respective excited states. 
Mostly interesting is the study of \cascade baryons due to the rare information about their excited states.
The comparison of these excited states to the better known excited spectrum of nucleons should offer a deeper understanding of the inner structure of baryons.

\noindent A strategy of studying excited \cascade baryons in antiproton-proton collisions based on an example will be presented.



