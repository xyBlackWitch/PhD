%%%%%%%%%%%%%%%%%%%%%%%%%%%%%%%%%%%%%%
% SAMPLE:TEX
%%%%%%%%%%%%%%%%%%%%%%%%%%%%%%%%%%%%%%
\documentclass[fleqn,twocolumn,a4paper]{ikpar}
\usepackage{times,mathptm}
\pagestyle{empty}

\begin{document}
\parindent=0pt
\frenchspacing

\title{{\bf
Are the Darmstadt Modes Scissors Modes?}}
\author{D. Zawischa*
and J. Speth
}
\maketitle
The Darmstadt modes
\cite{r1}
have been searched for
because modes of the scissors
type had been predicted \cite{r3,r4},
and in spite of rather big
discrepancies between the findings and the predictions they
have been called scissors modes.
Results of microscopic calculations
essentially agree with each other and also with experimental
data. But their interpretation is not evident. This motivated
investigations of classical and semiclassical models \cite{r14,r17,r17a}.
\par
\medskip

The nuclear density is parameterized by a Fermi function with radius
$R(\theta,\phi)$
and the nuclear shape by
%
\begin{equation} R(\theta, \phi) =  R_0 \,\left(1+{\delta\over 2} \cos (2\theta) +{\delta
\over 6}\right)
\label{(13)}\end{equation}
which defines the nuclear deformation parameter $\delta$.
We assume that $N=Z$ only to keep the expressions simple;
for the same reason we only consider isovector modes.
Instead of restricting ourselves to pure rotations (described by the
rotational angle $\alpha$ and angular velocity $\dot\alpha$),
we include as a second degree of
freedom a quadru\-pole mode with distortion parameter $\chi$.
Denoting by ${\bf u}$ the displacement of a fluid element from
its equilibrium position, we assume
the velocity of a fluid element of proton matter at
${\bf  r}$ to be
\begin{equation}
{\bf \dot u}({\bf r},t)={\bf  \dot r}({\bf  r},t)
= \dot\alpha\,(0,-z,y) + \dot\chi\,(0,z,y).
\label{(2)}\end{equation}
In any eigenmode of the nucleus $\alpha$ and $\chi$ are proportional, so
$\alpha=\alpha_0\,Q,\quad \chi=\chi_0\,Q
$
with the normal
coordinate $Q$.
\par
The kinetic energy of the
motion is obtained as in the classical case.
The contributions of protons and
neutrons are equal for $N=Z$, and their sum is simply
\begin{equation}
T={1\over 2} \left (\dot\alpha^2 I_1 + 2\dot\alpha\dot\chi (I_3-I_1)
+ \dot\chi^2 I_1\right )
\label{(9)}\end{equation}
where $I_1$ is the total rigid body
moment of inertia around the $x$-axis, $I_3$ that
around the $z$-axis.\par
The restoring force is derived from the asymmetry energy which
can be written \cite{r7,r20} in the form of an integral involving
$\delta\rho = \delta\rho^{\rm p} -\delta\rho^{\rm n}$
We have for small amplitude motion $\delta\rho
=-{\bf  u}
\cdot \nabla (\rho^{\rm p}+\rho^{\rm n}) %.\label{(11)}
$
which finally yields the restoring potential in the form
\begin{equation}V_{\rm asy}(\alpha,\chi)=
{1\over 2}C[\chi
-\delta^\prime\alpha]^2\label{(17)}\end{equation}
where  $\delta'=\delta/(1+\delta/6)\approx\delta$.
This result is remarkable:
the two modes which we consider are coupled
with each other and are therefore not the normal modes of the system,
and, moreover, one of the normal modes can be obtained by inspection:
If rotation and vibration occur
in the special mixture given by
\begin{equation} \chi=\delta^\prime\,\alpha , \label{(18)}\end{equation}
from eq. (\ref{(17)}) we see that
there is no restoring force. The reason is that the flow is tangential
to the nuclear surface,
and no separation of protons and neutrons occurs.
Thus classically
one of the normal modes is a zero energy mode which can not be excited.
\par
\begin{figure}[h]
\vskip 9cm
\caption{Streamline picture of the
current distribution of the zero energy
eigenmode of the classical nonspherical drop.
Full lines: protons, dotted lines: neutrons (or vice versa).
}
\end{figure}
Up to now we have completely ignored the fact that
the moving fluid is made up of fermions.
In a vibrating nucleus, the
time dependent velocity field distorts the many-body wave function.
In order to always remain in the lowest energy state
at any deformation, particles occupying
single particle states which move upwards in the Nilsson level scheme
would have to make transitions to energetically more favourable states.
It is assumed that in the nuclear fluid the collective
motion is so fast that these transitions do not occur \cite{Bertsch}.
To account for this in an
averaged way the Thomas-Fermi approximation can be applied.
This concept has been worked out in detail by Holzwarth et al.
\cite{r15}.
\par
\begin{table*}
\caption{
Energies and magnetic dipole strengths for the two
normal modes in different models but with the same set
of parameters. In the rigid rotor model, the high energy mode is
impossible.
The second row refers to the model without quantum corrections,
the data being obtained from eqs. (\protect\ref{(34)}) to
(\protect\ref{(40)})
with $D=0$, whereas for the NFD results in the last line
the value of $D$ determined by eq. (\protect\ref{(30)}) has been used.}
\par\bigskip
\hfill\vbox{{\halign
{\quad#\quad\hfil&\quad\hfil#\hfil\quad&\quad\hfil#\hfil\quad
&\quad\hfil#\hfil\quad&\quad\hfil#\hfil\quad\cr
\noalign{\smallskip\hrule\smallskip}
&\multispan2 \hfil Low energy mode\hfil& \multispan2\hfil
High energy mode\hfil\cr
\noalign{\smallskip\hrule\smallskip}
&$\hbar\omega$ (MeV)&$B$(M1)$\uparrow$ ($\mu_{\rm N}^2$)
&$\hbar\omega$ (MeV)&$B$(M1)$\uparrow$ ($\mu_{\rm N}^2$)\cr
\noalign{\smallskip\hrule\smallskip}
Rigid Rotors Model: &5.05&20.1&---&---\cr
Classical model: &0&0&20.6&4.94\cr
Nuclear Fluid Dyn. &2.59&10.0&23.8&3.19\cr
\noalign{\smallskip\hrule\smallskip}
}}}\hfill\break\par
\end{table*}
One obtains $\Delta T_{\rm F}$ and from this
\begin{equation}
V_{\rm elast}(\alpha,\chi)=\Delta T_{\rm F}=
{1\over 2}D\, \chi^2.\label{(30)}\end{equation}
where the force constant $D$ is defined. \par
Proceeding in standard fashion we get the equations of motion.
Putting
\begin{equation}\pmatrix{\alpha\cr\chi\cr}=
\pmatrix{\alpha _0\cr\chi _0\cr}\,Q_0\,e^{i\omega t}
\label{(32)}\end{equation}
we get a set of eigenvalue equations.
The eigenfrequencies are\begin{equation}\omega _1^2\approx {D\over I_1}
\cdot{{\delta^{\prime 2}}\over{1+{D/C}}}
\qquad\hbox{and}\qquad\omega_2^2\approx {{C+D}\over{I_1}}
\label{(34)}\end{equation}
Thus, since $D\ne 0$, the low frequency mode now has a finite frequency.
The corresponding normal modes are
\begin{equation} (\alpha_0,\chi_0)=(\alpha_0,\,\,\delta^\prime C(C+D)^{-1}\cdot\alpha_0)
\label{(35)}\end{equation}
and
\begin{equation} (\alpha_0,\chi_0)=
(\delta^\prime D(C+D)^{-1}\cdot \chi_0,\,\,\chi_0)\label{(35b)}\end{equation}
for the low energy and
for the high energy mode, respectively, where we have arbitrarily normalized
such that the larger component is 1. From (\ref{(34)}, \ref{(35)}) and
(\ref{(35b)}), by putting
$D=0$, we recover the former results of the classical model.
\par
The normal vibrations can be quantized in the usual way, to obtain
the transition strength to
\begin{equation}
B({\rm M}1)\!\!\uparrow\,
={{3I_1\hbar\omega}\over{16\pi\hbar^2}}{{(\alpha_0-\delta^\prime\chi_0)^2}
\over{\alpha_0^2+\chi_0^2-2\delta^\prime\alpha_0\chi_0}}
\label{(40)}\end{equation}
This formula encompasses the rigid rotor model (setting $\alpha_0=1$,
$\chi_0=0$).
\par
\par
We use $A=164$, $\delta=0.256$,
and the other parameters exactly as in \cite{r20}.
In the evaluation of the moments of inertia assume
a sharp nuclear surface.
We get $D/C=0.36$, and the energies and transition strengths as given
in the last row of table 1.
\par
We get essentially the same proportionality of the $B({\rm M}1)$-value
and excitation energy as in the two rotor model, so
too much transition
strength is obtained by a factor of $\approx 2$ (if we compare with
the total strength at low energy). In the microscopic calculation,
pairing strongly quenches the strength.
Pairing effects are missing in
our semiclassical treatment.
\par
The flow pattern of the low energy
mode differs only very slightly from
that of the classical zero mode, fig. 1, cf. eq. (\ref{(18)}) and (\ref{(35)}).
It is remarkable that it exhibits some of the features
which have been found in the transition densities calculated
microscopically \cite{r14,r13},
viz. that at the surface the flow is almost tangential.
It is not small in the inside of the nucleus.
The M1 form factor derived
from this current density deviates only
very little from the corresponding rigid rotor result, so that
there is almost no improvement of the agreement with experimental
data.
\par
\par\bigskip
%----------------------------------------------------------------

\begin{thebibliography}{99}
\bibitem{r1} D. Bohle, A. Richter, W. Steffen, A.E.L. Dieperink, N. Lo Iudice,
F. Palumbo, and O. Scholten, Phys. Lett. B {\bf 137}, 27 (1984)
\bibitem{r3} N. Lo Iudice and F. Palumbo, Phys. Rev. Lett. {\bf 41}, 1532 (1978);
 Nucl. Phys. A326, 193 (1979);
\bibitem{r4} R.R. Hilton,
Z. Phys. A {\bf 316}, 121 (1984) and references cited there
\bibitem{r14} D. Zawischa and J. Speth, Z. Physik {\bf A 339}, 97 (1991)
\bibitem{r17} D. Zawischa and J. Speth, Proc. of the Workshop on Nuclear
Shapes and Nuclear Structure at Low Excitation Energy, Carg
June 3 -- 7, 1991. Plenum, 1992;
\bibitem{r17a} D. Zawischa and J. Speth,
to be published in Nuclear Physics A (Proceedings of the International
Nuclear Physics Conference, Wiesbaden 1992)
\bibitem{r7} R. Nojarov, Z. Bochnacki, and A. Faessler,
 Z. Phys. A {\bf 324}, 289 (1986)
\bibitem{r20} N. Lo Iudice, F. Palumbo, A. Richter, and H.J. W\"ortche,
Phys. Rev. C {\bf 42}, 241 (1990)
\bibitem{Bertsch} G.F. Bertsch, Ann. Phys. (N.Y.) {\bf 86}, 138 (1974)
\bibitem{r15} G. Holzwarth and G. Eckart, Z. Physik {\bf A 284}, 291 (1978)%
, Nucl. Phys. {\bf A325}, 1 (1979)
\bibitem{r13} J. Speth  and D. Zawischa, Phys. Lett. B {\bf 211}, 247 (1988)
and B {\bf 219}, 529 (1989)
\end{thebibliography}
\medskip
* Institut f\"ur Theoretische Physik,
Universit\"at Hannover,
W-3000 Hannover, Germany\par
\vfill
\end{document}

